\section{Exercise 1}

%\usepackage{indentfirst}
%\setlength{\parindent}{2em} 
\paragraph*{1}
I shorten the notation by using sin as a replacement for sin(x) and cos as a replacement for cos(x).
We use the binet formula to get the power of $A^n$.
We first estimate the eigenvectors, then concatenate them into matrix $P$ and use this matrix to compute the diagonal matrix $B$, which can be used to directly take the power for the matrix $A$. Afterwards we need to reverse the operations, which we had done on achieving $B$ to get $A^n$.
\begin{gather*}
A^n =
\left( \begin{array}{cc}
\cos & \sin \\
-\sin & \cos \\
\end{array} \right)^n , v = \left( \begin{array}{c} x_1\\x_2 \end{array} \right), ( A-\lambda I )v = 0
\\
\left( \begin{array}{cc}
\cos & \sin \\
- \sin & \cos 
\end{array}
\right) -
\left( \begin{array}{cc}
\lambda & 0 \\
0 & \lambda
\end{array} 
\right)
= \\
(\cos ^2 - \lambda) + \sin ^2 = 0 \\
\lambda _1 = \ii \sin + \cos \\
\lambda _2 = - \ii \sin + \cos
\end{gather*}
Now we achieved the eigenvalues, so we can estimate the eigenvectors.
The eigenvector $\lambda _1$
\begin{gather*}
\left( \begin{array}{cc}
\cos & \sin \\
- \sin & \cos \\
\end{array} 
\right)
-
\left( \begin{array}{cc}
\ii \sin + \cos & 0\\
0 & \ii \sin + \cos 
\end{array} 
\right)
=
\left( \begin{array}{c}
\ii \\
1 
\end{array}
\right)
\end{gather*}
Symmetrically the vector for $\lambda _2$ is:
\begin{gather*}
\left( \begin{array}{c}
-\ii \\
1
\end{array} \right)
\end{gather*}
The matrix $P$ is just a concatenation of $\lambda _1$ and $\lambda _2$.
\begin{equation}
P = \left( \begin{array}{cc}
\ii & -\ii \\
1 & 1
\end{array} \right)
\end{equation}
To calculate an diagonal matrix, we use the equation $B=P^{-1}AP$ to transform A into a diagonal matrix.
\begin{gather*}
P^{-1} = \dfrac{1}{\text{det}(P)} \text{adj}(P) =\\
\dfrac{1}{2\ii}
\left( \begin{array}{cc}
1 & \ii \\
-1 & \ii
\end{array}
\right)
= 
\left( \begin{array}{cc}
\dfrac{1}{2\ii} & \dfrac{1}{2} \\
- \dfrac{1}{2\ii} & \dfrac{1}{2}
\end{array}
\
\right)
\\
P^{-1}AP =
\left( \begin{array}{cc}
\dfrac{1}{2\ii} & \dfrac{1}{2} \\
- \dfrac{1}{2\ii} & \dfrac{1}{2}
\end{array}
\right)
\left( \begin{array}{cc}
\cos & \sin \\
-\sin & \cos
\end{array}
\right)
\left( \begin{array}{cc}
\ii & -\ii \\
1 & 1
\end{array}
\right)
=\\
\left( \begin{array}{cc}
\left( \dfrac{cos}{2\ii} - \dfrac{\sin}{2} \right)\ii + \dfrac{\sin}{2 \ii} + \dfrac{\cos}{2} &
\left( \dfrac{cos}{2 \ii} - \dfrac{\sin}{2} \right)(-\ii) + \dfrac{\sin}{2 \ii}+\dfrac{\cos}{2} \\
\left( - \dfrac{\cos}{2 \ii} - \dfrac{\sin}{2} \right)\ii + -\dfrac{\sin}{2 \ii} + \dfrac{\cos}{2} &
\left( - \dfrac{\cos}{2 \ii} - \dfrac{\sin}{2} \right)(-\ii) - \dfrac{\sin}{2 \ii} + \dfrac{\cos}{2}
\end{array}
\right)
=\\
\left( \begin{array}{cc}
\cos - \ii \sin & 0\\
 0 & \cos + \ii \sin 
\end{array}
\right)
\end{gather*}
Now we can apply the power operation on the matrix.
\begin{equation*}
B^n = \left( \begin{array}{cc}
(\cos - \ii \sin)^n & 0\\
 0 & (\cos + \ii \sin )^n
\end{array}
\right)
\end{equation*}
Now we can apply the matrix exponential onto $A$, since $A = PB^{n}P^{-1}$.
\begin{gather*}
\left( \begin{array}{cc}
\ii & - \ii \\
1 & 1
\end{array}
\right)
\left( \begin{array}{cc}
(\cos - \sin \ii)^n & 0 \\
0 & (\cos + (\ii \sin )^n
\end{array}
\right)
\left( \begin{array}{cc}
\dfrac{1}{2\ii} & \dfrac{1}{2} \\
-\dfrac{1}{2 \ii} & \dfrac{1}{2}
\end{array}
\right)
=\\
\left( \begin{array}{cc}
(\cos - \sin \ii)^n \ii & (\cos + \ii \sin)^n)(-\ii) \\
(\cos - \sin \ii)^n & (\cos + \ii \sin)^n 
\end{array}
\right)
\left( \begin{array}{cc}
\dfrac{1}{2\ii} & \dfrac{1}{2} \\
-\dfrac{1}{2 \ii} & \dfrac{1}{2}
\end{array}
\right)
=\\
\dfrac{1}{2}
\left( \begin{array}{cc}
\left( \cos - \sin \ii \right)^n + (\cos + \sin \ii )^n & (\cos - \sin)^n \ii - (\cos + \sin \ii)^n \ii \\ 
\ii (\cos + \sin)^n  - \ii (\cos - \sin \ii )^n & \left( \cos - \sin \ii \right)^n + (\cos + \sin \ii )^n 
\end{array}
\right)
=
A^n
\end{gather*}

After some basic mathematics procedures we get the simplified matrix as shown below.

\begin{gather}
A^n = 
\left( 
	\begin{array}{cc}
		\cos(n\cdot x) & \sin(n\cdot x)\\
		-\sin(n\cdot x) & \cos(n\cdot x)
	\end{array}
\right)
\end{gather}

\paragraph*{2}
Like in exercise 1 we begin by getting the eigenvalues and eigenvectors and the matrix $P$.
\begin{gather*}
A =\left( \begin{array}{cc}
1 & 1 \\ -1 & 1
\end{array} \right)
\text{Eigenvalues of } A:
\left( \begin{array}{cc}
1 - \lambda & 1 \\ 
-1 & 1 -\lambda 
\end{array} \right)
=
(1-\lambda)^2 + 1 = 0\\
\lambda _1 = 1 +\ii \\
\lambda _2 = 1 -\ii \\
\text{Eigenvector of }\lambda _1, \lambda _2
\left[
\left( \begin{array}{cc}
1 & 1 \\
-1 & 1
\end{array} \right) 
-
\left( \begin{array}{cc}
1+\ii & 0\\
0 & 1+\ii
\end{array} \right)
\right]
\left( \begin{array}{c}
x_1 \\ x_2
\end{array} \right)
= 0 \\
v_1 =
\left( \begin{array}{c}
1 \ii
\end{array} \right)
\\
v_2 =
\left( \begin{array}{c}
1 \\ -\ii
\end{array} \right)
\end{gather*}
Since we computed the eigenvectors $v_1$ and $v_2$, we continue by calculating the diagonal matrix $B$.
\begin{gather*}
P =
\left( \begin{array}{cc}
1 & 1\\
\ii & -\ii
\end{array} \right) 
\Rightarrow P^{-1} = \dfrac{1}{\text{det}(P)}
\left( \begin{array}{cc}
-\ii & -1 \\
-\ii & 1
\end{array} \right) =  \dfrac{1}{-2 \ii}
\left( \begin{array}{cc}
-\ii & -1 \\
-\ii & 1
\end{array} \right) = 
\left( \begin{array}{cc}
\dfrac{1}{2} & \dfrac{1}{2\ii}\\
\dfrac{1}{2} & -\dfrac{1}{2\ii}
\end{array} \right)\\
B=
\left( \begin{array}{cc}
\dfrac{1}{2} & \dfrac{1}{2\ii}\\
\dfrac{1}{2} & -\dfrac{1}{2\ii}
\end{array} \right)
\left( \begin{array}{cc}
1 & 1\\
-1 & 1
\end{array} \right)
\left( \begin{array}{cc}
1 & 1\\
\ii & -\ii
\end{array} \right)
=
\left( \begin{array}{cc}
\dfrac{1}{2} & \dfrac{1}{2\ii}\\
\dfrac{1}{2} & -\dfrac{1}{2\ii}
\end{array} \right)
\left( \begin{array}{cc}
1+\ii & 1-\ii \\
-1+\ii & -1-\ii
\end{array} \right) 
=
\left( \begin{array}{cc}
1+\ii & 0\\
0 & 1-\ii
\end{array} \right)
\\
B^n =
\left( \begin{array}{cc}
(1+\ii)^n & 0\\
0 & (1-\ii)^n
\end{array} \right)
\\
A^n=PB^{n}P^{-1} =
\left( \begin{array}{cc}
1 & 1 \\
\ii & -\ii
\end{array} \right)
\left( \begin{array}{cc}
(1+\ii)^n & 0\\
0 & (1-\ii)^n
\end{array} \right)
\left( \begin{array}{cc}
\dfrac{1}{2} & \dfrac{1}{2\ii}\\
\dfrac{1}{2} & -\dfrac{1}{2\ii}
\end{array} \right) 
=
\left( \begin{array}{cc}
1 & 1 \\
\ii & -\ii
\end{array} \right)
\left( \begin{array}{cc}
\dfrac{(1+\ii)^n}{2} & \dfrac{(1+\ii)^n}{2\ii} \\
\dfrac{(1-\ii)^n}{2} & -\dfrac{(1-\ii)^n}{2\ii}
\end{array} \right)
=\\
\dfrac{1}{2}
\left( \begin{array}{cc}
(1+\ii)^n +(1-\ii)^n & (1+\ii)^n - (1-\ii)^n \\
\ii (1+\ii)^n - \ii (1-\ii)^n & (1+\ii)^n + (1-\ii)^n
\end{array} \right) = A^n
\end{gather*}

\paragraph*{3}
To solve these we do not use the Methods as in Exercise 1 or 2 because the Eigenvector of these Matrix is:
\begin{gather*}
V_1=\left( \begin{array}{c}
	1 \\
	0 \\
	0 \\
	0 \\
	0 \\
\end{array} \right)
\end{gather*}

With these vector it is not possible to make the transformations as shown at the previous paragraphs.
After calculating a few powers of the Matrix we got these series:
\begin{gather*}
\begin{array}{cc}
A^2=
	\left( \begin{array}{ccccc}
		a^2 &  2*a&    1&    0&    0\\
		0&  a^2&  2*a&    1&    0 \\
		0&    0&  a^2&  2*a&    1\\
		0&    0&    0&  a^2&  2*a \\
		0&    0&    0&    0&  a^2
	\end{array} \right)
	&
A^3=
	\left( \begin{array}{ccccc}
		a^3& 3*a^2&   3*a&     1&    0\\
		0&   a^3& 3*a^2&   3*a&     1\\
		0&     0&  a^3& 3*a^2&   3*a\\
		0&     0&    0&   a^3& 3*a^2\\
		0&     0&    0&     0&   a^3
	\end{array} \right)
	\\ \\
A^4=
	\left( \begin{array}{ccccc}
		a^4& 4*a^3& 6*a^2&  4*a&    1\\
		0&   a^4 & 4*a^3& 6*a^2&   4*a\\
		0&     0&   a^4& 4*a^3& 6*a^2\\
		0&     0&     0&   a^4& 4*a^3\\
		0&     0&     0&     0&   a^4
	\end{array} \right)
	&
A^5=
	\left( \begin{array}{ccccc}
		a^5& 5*a^4& 10*a^3& 10*a^2&    5*a\\
		0&   a^5&  5*a^4& 10*a^3& 10*a^2\\
		0&     0&    a^5&  5*a^4& 10*a^3\\
		0&     0&      0&    a^5&  5*a^4\\
		0&     0&      0&      0&    a^5
	\end{array} \right)
\end{array}
\end{gather*}

After looking for a longer time at these equations It is totally obvious the Matrix has to be:

\begin{gather*}
	\begin{medsize}
	A^n=
		 \begin{pmatrix}
		a^n& a^{-1 + n}\cdot n& \dfrac{1}{2} a^{-2 + n}\cdot (-1 + n)\cdot n& \dfrac{1}{6} a^{-3 + n}\cdot (-2 + n)\cdot (-1 + n)\cdot n& \dfrac{1}{24} a^{-4 + n}\cdot (-3 + n)\cdot (-2 + n)\cdot (-1 + n)\cdot n\\
		0& a^n& a^{-1 + n}\cdot n& \dfrac{1}{2} a^{-2 + n}\cdot (-1 + n)\cdot n& \dfrac{1}{6}\cdot a^{-3 + n}\cdot (-2 + n)\cdot (-1 + n)\cdot n\\
		0& 0& a^n& a^{-1 + n}\cdot n&  \dfrac{1}{2} a^{-2 + n}\cdot (-1 + n)\cdot n\\
		0& 0& 0& a^n& a^{-1 + n}\cdot n\\ 
		0& 0& 0& 0& a^n
		\end{pmatrix}
	\end{medsize}
\end{gather*}

\section{Exercise 2}

\paragraph*{1}
Computing $A^{-1}B$.
\begin{gather*} 
A^{-1} = \frac{1}{\text{det}(A)}
\left( \begin{array}{ccc}
4 & 5 & 0 \\
2 & 3 & 1 \\
2 & 7 & -3 \\
\end{array} \right)
=
- \frac{1}{24} 
\left( \begin{array}{ccc}
-16 & 15 & 5 \\
8 & -12 & -4 \\
8 & -18 & 2 \\
\end{array} \right)
\\
AB = 
\frac{1}{12} \left( \begin{array}{cccc}
12 & 0 & -30 & 95 \\
0 & 12 & 24 & -52 \\
0 & 0 & 0 & -46 
\end{array} \right)
\end{gather*}
\paragraph*{2}
Computing $CA^{-1}$
\begin{gather*}
CA^{-1} =
\left( \begin{array}{ccc}
4 & 5 & 0 \\
2 & 3 & 1\\
2 & 7 & 9\\
-2 & 3 & 7\\
\end{array} \right)
( -\frac{1}{24})
\left( \begin{array}{ccc}
-16 & 15 & 5 \\
8 & -12 & -4 \\
8 & -18 & 2 \\
\end{array} \right)
=
\frac{1}{3}
\left( \begin{array}{ccc}
3 & 0 & 0\\
0 & 3 & 0\\
-12 & 27 & 0\\
-14 & 24 & 1\\
\end{array} \right)
\end{gather*}

\section{Exercise 3}
We want to proof the following equation:
\begin{equation}
\label{ex3}
\adj (AB) = \adj (B) \adj (A)
\end{equation}
It is already known that $A \adj (A) = \deter (A) I$, so
\begin{equation}
\label{adjA}
\adj (A) = A^{-1} \deter (A) I
\end{equation} 
Also it is known that 
\begin{equation}
\label{detmul}
\deter (AB) = \deter(A) \deter(B)
\end{equation}
We substitute \ref{adjA} into \ref{ex3}, by using \ref{detmul}.
\begin{gather*}
\adj (B) \adj (A) = \deter (B) B^{-1} I \deter (A) A^{-1} I = \\
\deter (A) \deter (B) B^{-1} I A^{-1} I = \deter (AB) (AB)^{-1}
\end{gather*}
Using $C=AB$ we get:
\begin{gather*}
\deter(C) C^{-1} = \adj (C) \Rightarrow \adj (AB) = \adj (A) \adj (B)
\end{gather*}

\section{Exercise 4}
Assuming having a matrix $A$ with dimensions of $n\times m$.\\
We assume that there exists two ranks, rowrank(A) and colrank(A).
\begin{gather*}
A = \left( \begin{array}{ccc}
a_{11} & \hdots & a_{n1} \\
\vdots & \ddots & \vdots \\
a_{1m} & \hdots & a_{nm}
\end{array} \right)
\end{gather*}
If we modify $A$ to $A^{'}$ by doing Gaussian elimination, $A^{'}$ will have exactly $r$ non-zero rows, which will be denoted as $\text{rowrank}(A) = r$. 
By doing the transpose $A^{{'}^T}$, we can still observe that the rowrank is $r$ times non-zero rows. Since the columns in the transpose are the rows of the non-transpose, it leads to:
\begin{equation*}
\text{rowrank}(A^{'}) = \text{rowrank}(A^{{'}^T}) = \text{colrank}(A^{'}) = \text{colrank}(A^{{'}^T}) = \text{rowrank}(A) = \text{colrank} (A)
\end{equation*}

I show that this behaviour is working as well for the square $AA^T$ matrix. Since it is already proofed that the column rank is the row rank, we can use the basic formula for ranks :$\rank(AB) \leq \min(\rank (A), \rank(B))$.
\begin{equation}
\rank(A A^T) \leq \min( \rank (A) , \rank (A^T)) \Rightarrow \rank (A)=\rank (A^T)= \rank (A A^T)
\end{equation}

\section{Exercise 5}
To proof $\rank (A^n) = \rank (A^{n+1})$  we use recursion. 
\begin{equation}
\label{rankeq}
\rank (AB) \leq \min (\rank (A), \rank (B))
\end{equation}
By the use of \ref{rankeq} we get the following recursion:
\begin{gather*}
\rank (A^{n+1}) = \\ 
\rank (A^{n} A) \leq \min ( \rank (A^n), \rank(A) ) \\
\rank (A^n) = \rank (A^{n-1} A) \leq \min ( \rank (A^{n-1}), \rank(A) ) \\
\vdots \\
\rank(A A) = \min ( \rank (A), \rank(A) ) = \rank(A)
\end{gather*}
So after this recursion, it can be seen that $\rank (A^{n+1)}) = \rank (A^{n})$

\section{Exercise 6}
We use the Leibnitz formula to determinate the determinate.
\begin{equation}
\label{leib}
\det\begin{pmatrix}A& B\\ C& D\end{pmatrix} = \det(D) \det(A - B D^{-1} C)
\end{equation}
From equation \ref{leib}, we can rewrite the matrix $X$ given as:
\begin{gather*}
\begin{pmatrix}
0& x_1 & \hdots & x_n \\
-x_1 & a_{11} & \hdots & a_{1n}\\
\vdots & \vdots	& & \vdots\\
-x_n & a_{n1} & \hdots & a_{nn} 
\end{pmatrix}  \\
\det(A) \det(0 - x A^{-1} (-x^T)) =\\
\det(A) \det( x \frac{1}{\det(A)} \adj(A) x^T) =\\
\det(x*\adj(A)*x^{T}) = | x*\adj(A)*x^{T} |
\end{gather*}

\section{Exercise 7}
\paragraph*{a}
The measure matrix as the inner product is given as: 
\begin{gather*}
\{ f(t) = a_0 + a_1 t + a_2 t^2 : a_0,a_1,a_2 \in R \}\\
1 \rightarrow a_0 - a_1 = 1-t^2 \\
t \rightarrow a_1 - a_2 = t-1 \\
t^2 \rightarrow a_2 - a_0 = t^2 -t \\
\end{gather*}
To get the measure matrix we concatenate:
\begin{gather*}
1 \rightarrow \left( 1 , t , t^2 \right)
\left(
\begin{array}{c}
1 \\ 0 \\ 1
\end{array} \right) \\
t \rightarrow \left( 1 , t , t^2 \right)
\left(
\begin{array}{c}
-1 \\ 1 \\ 0
\end{array} \right) \\
t^2 \rightarrow
\left( 1 , t , t^2 \right)
\left(
\begin{array}{c}
0 \\ -1 \\ 1
\end{array} \right)
\end{gather*}
The measure matrix:
\begin{gather*}
\left( 1 , t , t^2 \right)
\left( \begin{array}{ccc}
1 & -1 & 0\\
0 & 1 & -1 \\
1 & 0 & 1 \\
\end{array} \right)
\left( \begin{array}{c}
a_0\\ a_1 \\ a_2
\end{array} \right)
\end{gather*}

\paragraph*{b}