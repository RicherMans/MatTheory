\section{Exercise 1}

%\usepackage{indentfirst}
%\setlength{\parindent}{2em} 
\paragraph*{1}
I shorten the notation by using sin as a replacement for sin(x) and cos as a replacement for cos(x).
We use the binet formula to get the power of $A^n$.
We first estimate the eigenvectors, then concatenate them into matrix $P$ and use this matrix to compute the diagonal matrix $B$, which can be used to directly take the power for the matrix $A$. Afterwards we need to reverse the operations, which we had done on achieving $B$ to get $A^n$.
\begin{eqnarray*}
\left( \begin{array}{cc}
\cos & \sin \\
-\sin & \cos \\
\end{array} \right)^n , v = \left( \begin{array}{c} x_1\\x_2 \end{array} \right), ( A-\lambda I )v = 0
\\
\left( \begin{array}{cc}
\cos & \sin \\
- \sin & \cos 
\end{array}
\right) -
\left( \begin{array}{cc}
\lambda & 0 \\
0 & \lambda
\end{array} 
\right)
= \\
(\cos ^2 - \lambda) + \sin ^2 = 0 \\
\lambda _1 = \ii \sin + \cos \\
\lambda _2 = - \ii \sin + \cos
\end{eqnarray*}
Now we achieved the eigenvalues, so we can estimate the eigenvectors.
The eigenvector $\lambda _1$
\begin{eqnarray*}
\left( \begin{array}{cc}
\cos & \sin \\
- \sin & \cos \\
\end{array} 
\right)
-
\left( \begin{array}{cc}
\ii \sin + \cos & 0\\
0 & \ii \sin + \cos 
\end{array} 
\right)
=
\left( \begin{array}{c}
\ii \\
1 
\end{array}
\right)
\end{eqnarray*}
Symmetrically the vector for $\lambda _2$ is:
\begin{eqnarray*}
\left( \begin{array}{c}
-\ii \\
1
\end{array} \right)
\end{eqnarray*}
The matrix $P$ is just a concatenation of $\lambda _1$ and $\lambda _2$.
\begin{equation}
P = \left( \begin{array}{cc}
\ii & -\ii \\
1 & 1
\end{array} \right)
\end{equation}
To calculate an diagonal matrix, we use the equation $B=P^{-1}AP$ to transform A into a diagonal matrix.
\begin{eqnarray*}
P^{-1} = \frac{1}{\text{det}(P)} \text{adj}(P) =\\
\frac{1}{2\ii}
\left( \begin{array}{cc}
1 & \ii \\
-1 & \ii
\end{array}
\right)
= 
\left( \begin{array}{cc}
\frac{1}{2\ii} & \frac{1}{2} \\
- \frac{1}{2\ii} & \frac{1}{2}
\end{array}
\
\right)
\\
P^{-1}AP =
\left( \begin{array}{cc}
\frac{1}{2\ii} & \frac{1}{2} \\
- \frac{1}{2\ii} & \frac{1}{2}
\end{array}
\right)
\left( \begin{array}{cc}
\cos & \sin \\
-\sin & \cos
\end{array}
\right)
\left( \begin{array}{cc}
\ii & -\ii \\
1 & 1
\end{array}
\right)
=\\
\left( \begin{array}{cc}
\left( \frac{cos}{2\ii} - \frac{\sin}{2} \right)\ii + \frac{\sin}{2 \ii} + \frac{\cos}{2} &
\left( \frac{cos}{2 \ii} - \frac{\sin}{2} \right)(-\ii) + \frac{\sin}{2 \ii}+\frac{\cos}{2} \\
\left( - \frac{\cos}{2 \ii} - \frac{\sin}{2} \right)\ii + -\frac{\sin}{2 \ii} + \frac{\cos}{2} &
\left( - \frac{\cos}{2 \ii} - \frac{\sin}{2} \right)(-\ii) - \frac{\sin}{2 \ii} + \frac{\cos}{2}
\end{array}
\right)
=\\
\left( \begin{array}{cc}
\cos - \ii \sin & 0\\
 0 & \cos + \ii \sin 
\end{array}
\right)
\end{eqnarray*}
Now we can apply the power operation on the matrix.
\begin{equation*}
B^n = \left( \begin{array}{cc}
(\cos - \ii \sin)^n & 0\\
 0 & (\cos + \ii \sin )^n
\end{array}
\right)
\end{equation*}
Now we can apply the matrix exponential onto $A$, since $A = PBP^{-1}$.
\begin{eqnarray*}
\left( \begin{array}{cc}
\ii & - \ii \\
1 & 1
\end{array}
\right)
\left( \begin{array}{cc}
(\cos - \sin \ii)^n & 0 \\
0 & (\cos + (\ii \sin )^n
\end{array}
\right)
\left( \begin{array}{cc}
\frac{1}{2\ii} & \frac{1}{2} \\
-\frac{1}{2 \ii} & \frac{1}{2}
\end{array}
\right)
=\\
\left( \begin{array}{cc}
(\cos - \sin \ii)^n \ii & (\cos + \ii \sin)^n)(-\ii) \\
(\cos - \sin \ii)^n & (\cos + \ii \sin)^n 
\end{array}
\right)
\left( \begin{array}{cc}
\frac{1}{2\ii} & \frac{1}{2} \\
-\frac{1}{2 \ii} & \frac{1}{2}
\end{array}
\right)
=\\
\frac{1}{2}
\left( \begin{array}{cc}
\left( \cos - \sin \ii \right)^n + (\cos + \sin \ii )^n & (\cos - \sin)^n \ii - (\cos + \sin \ii)^n \ii \\ 
\ii (\cos + \sin)^n  - \ii (\cos - \sin \ii )^n & \left( \cos - \sin \ii \right)^n + (\cos + \sin \ii )^n 
\end{array}
\right)
=
A^n
\end{eqnarray*}

\section{Exercise 2}